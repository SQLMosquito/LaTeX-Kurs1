%!TEX TS-program=pdflatex
%!TEX encoding=latin1
%%%%%%%%
% The text in this file is copyright (c) 2011 by Sara Eliasson
% All rights reserved.
%%%%%%%%
% The LaTeX code, that is part of the  introduction to LaTeX by Christian Mueller 
% and Nicolas Zahn, is licensed under the Creative Commons Attribution-ShareAlike 3.0 
% Unported License. To view a copy of this license, visit
% http://creativecommons.org/licenses/by-sa/3.0/ or send a letter to 
% Creative Commons, 444 Castro Street, Suite 900, Mountain % View, California, 94041, USA.
%%%%%%%%
\documentclass[fontsize=11pt]{scrartcl}

\usepackage[os=win,lang=en]{IPZstyle}

\usepackage{csquotes} % erweiterte Zitierm�glichkeiten
\usepackage{setspace} % Zeilenabst�nde einfach �ndern

%% Schriften �ndern %%
\usepackage{lmodern} % sch�nere Version der LaTeX Standardschrift
% \usepackage{bookman}
% \usepackage{avant}

%% Einzug erste Zeile %%
\setlength{\parindent}{1.5em} % Einzug erste Zeile erh�hen

%% Titelinformationen %%
\title{Comparing socialization- and diffusion-theory in the context of
  international organizations}
% \subtitle{Untertitel}
\course{Role of international organizations}
\date{June 18, 2011}
\lecturer{Dr. Gabriele Spilker}
\term{Spring semester 2011}
\type{Research paper}
\author{Sara Helena Eliasson}
\authorinfo{Matrikel-No.: 08-736-621\\
Email: sara.eliasson@uzh.ch\\
Landline: +41 43 268 40 88\\
Siewerdtstrasse 10\\
CH-8050 Z�rich}

%%%%%%%%%%%%%%%%%%%%%%%%%%%%%
% hier beginnt das Dokument %
%%%%%%%%%%%%%%%%%%%%%%%%%%%%%

\begin{document}

\pagenumbering{roman} % Seitennummerierung: r�mische Zahlen

\maketitle % Titelseite

\tableofcontents % Inhaltsverzeichnis

\clearpage % neue Seite
\onehalfspacing % schaltet den 1.5-fachen Zeilenabstand ein
\setcounter{page}{1} % setzt die Seitenzahlen auf 1
\pagenumbering{arabic} % Seitennummerierung: arabische Zahlen


\section{Introduction}
\label{sec:intro}

Modern, globalized life in the 21\textsuperscript{st} century is influenced by
very complex processes, but finding out how ordinary citzens are influenced by
international Organizations (IO's) is very difficult. The shaping of new
international treaties, policies and practices and the impact they have on us
is a research field shared by the big social sciences: sociology, political
science and psychology among others. This research paper is focused on
discussing the process of transferring ideas, norms and practices both
vertically between different levels of society and horizontally between
geographical entities. 


During the lecture \enquote{International Organizations -- role of international
organizations in world politics} many big themes and questions were brought
up. Some of them were: Who and what lies at the core of an organization, unitary
actors or combinations of individuals? How much power do they have, what kind of
organization is more successful? How can they increase their success rate by
using different instruments of soft and hard law? (Re-)answering some of these
questions is the second goal of this paper.


Because most of the big IO's have evolved into heavy normative organizations
they not only try to reach some practical, but also normative goals, i.e.
strengthening Human Rights, liberalizing world trade or promoting general peace
or cooperation.  Looking into the process of promoting these norms is very
interesting. Then more questions arise: Can IO's promote peace?  How much
influence do they have on the behavior of their members and, in some extent,
even their non-members? Discussing these questions is the primary goal of this
paper. 


Two theories will be examined to analyze this: Socialization- and diffusion
theory. First a summary of history and development will be given, followed by a
general overview of both theories. Then their place in modern research will be
reviewed. The last chapters adopt a more comparative view and look into some of
the advantages and disadvantages of both theories. 

\section{Theory}
\label{sec:theory}

Socialization-theory applies to different entities in the social sciences. It is
foremost used when analyzing individual behavior, but in this research paper
the goal is not individual socialization, but rather socialization in its
international perspective. What the author would like to find out is how
socialization-theory on an aggregate level works and how collectives such as
countries can be socialized into different behavior. But before doing this, we
have to return to the individual with a definition about what individual
socialization in a sociological context means and then work ourselves up to
aggregate level.


\blockquote[Hooghe 2005: 865; square brackets inserted by the
author][.]{Socialization refers to the process of inducing individuals into the
norms and rules of a given community. The mechanisms by which this occurs may
range from the self-cautious (for example normative [per-]suasion) to the
subconscious (for example social mimicking or role playing), and from the
instrumental (for example shaming) to the non- instrumental (for example
communication)}


The same principle will be used in some extent for section~\ref{sec:diffusion}
when analyzing diffusion processes.


\subsection{Socialization theory}
\label{sec:socialization}

In a political science context, socialization research started out as a result
of growing interest in political socialization (Wilson 1981). This new research
field tried to figure out where political motivations of individuals and
politicians were originating from. It also tried to understand the founding of
values, attitudes, practices and, crucially, moral judgment and moral behavior.
Jack Dennis (1968) summarized the research results of his time, and designed a
scheme of factors influencing behavior and political socialization, different
from the ones used by sociologist and psychologists. It highlighted factors of
the political system and different political stages and processes which could
lead to changes in political socialization-outcomes between countries.
Integration and identification with the political system in the native country
were two especially important factors. 


According to Martha Finnemore \& Kathryn Sikkink (1998), interest in
socialization got lost during the popularity rise of rational choice and
economic modeling of behavior in the 1960's and 70's, but became relevant
again in the 80's. Scientists from many different research areas found again
that it was not enough to consider changes in attitude or identity in
individuals or countries only as externalities (\emph{positive} or
\emph{negative}) or as \enquote{spill-over effects}. Also, research in norms
changes and normative behavior attracted more attention because of the increase
in two things: first systematic empirical results and globalization. The speed
of globalization led to the question if norms promoted stability or if they
promoted change (Finnemore \& Sikkink 1998).


Now, some definitions of norms might be in place. The first is a standard
definition and the second is from Martha Finnemore \& Kathryn Sikkink (1998:
888). For the rest of this section only the second and third aspect will be
considered, because of the interest in how they influence states and IO's. 


\begin{quote}
  \textbf{Norm}: \enquote{a standard of appropriate behavior for actors with a
    given identity}.
\end{quote}

\begin{quote}
  \textbf{3 aspects of norms}: their origin, the mechanisms by which they
  exercise influence, and the conditions under which norms will be influential
  in world politics
\end{quote}


Drawing on other political scientists, theoretical cooperation on the
international scene is the result of reciprocity (see Keohane 1986, Axelrod
1981 and others). Also the idea of the \enquote{shadow of the future} helps foster
non-defecting behavior, even under the state of anarchy. International
socialization-theorists argue that on top of utility-maximizing behavior and
reciprocity, order and stability on the international scene is created with the
use of norms (Dobbin \textit{et. al.} 2007). They limit
behavior, because they narrow down the possible ways of acting for players.
James Morrow (1994), highlights that this reduces informational problems and
that information about different preferences reduces uncertainty in
cooperation. Due to norms, countries cannot act as they like, because they may
have to fear retaliation and sanctions from other international actors. 


How new norms emerge can be seen in table~\ref{tab:stages-norms}. The process
includes 3 stages in the \enquote{lifecycle of norms}, which can be found in the
fields of related social sciences too (Finnemore \& Sikkink 1998: 895). At stage
1 are norm entrepreneurs trying to persuade states or leaders to implement new
ideas because of normative goals or views. The second stage is describing the
spreading of the norm through the population, hence the word \enquote{cascade}
where the norm moves like a \enquote{wave}. The last stage shows internalization
of the norm but also a rise in practitioners and experts dealing with it because
of its growing recognition as an existing problem.



\begin{table}[htp]
  \centering
  \begin{tabularx}{.95\textwidth}{p{2cm}>{\raggedright\arraybackslash}X>{\raggedright\arraybackslash}X>{\raggedright\arraybackslash}X}
    \toprule
    & \textbf{Stage 1:} & \textbf{Stage 2:} & \textbf{Stage 3:} \\
    & \textit{Norm emergence} & \textit{Cascade} & \textit{Internationalization} \\ \midrule
\textit{Actors} & Norm entrepreneurs with organizational platforms & States, IOs,
\textit{networks} & Law, professions, bureaucracy \\ \addlinespace
\textit{Motives} & Altruism, empathy, ideational, commitment & Legitimacy, reputation,
esteem & Conformity \\ \addlinespace
\textit{Dominant mechanisms} & Persuasion & Socialization, institutionalization,
demonstration & Habit, institutionalization\\
\bottomrule
  \end{tabularx}
  \caption{Stages of norms.}
  \label{tab:stages-norms}
\end{table}


Why states change behavior is then either because of domestic or international
pressure or both together. Judith Kelly (2004) shows these processes in action
in her analysis of Eastern European countries. In her paper, she identifies how
IO's can use their power to pressure countries to promote change. She concludes
that socialization-based efforts such as persuasion and using social influence
\emph{can be} effective on their own, but their efficiency is highly increased
when combined with incentives, for example membership-conditionality. She also
highlights the complex interplay between IO's, domestic policy-makers,
politicians and media. Finnemore and Sikkink (1998: 893) also supports this,
drawing attention to the interplay between national and international norms,
calling in a \enquote{two-level norm-game} in the words of Robert Putnam (1998).


\subsection{Diffusion theory}
\label{sec:diffusion}

Diffusion in its most general terminology is used in many fields of science to
describe the melting together of different substances or behavior. Elihu Katz
(1999) traces its origin back to the end of the 19\textsuperscript{th} century,
where it was first used in describing phenomenons of mass communication. In
political science today it means:


\blockquote[Nowlin 2011: 48][.]{Policy diffusion research tracks how similar
policy innovations are adopted across states \textelp{} or across countries in
comparative contexts}


In the current scientific debate not only this term is used, instead there are
many different theories about diffusion in the policy process. [For a summary
see Zachary Elkins and Beth Simmons (2005) or Matthew Nowlin (2011)]. But all
theories of diffusion build on the assumption that the world is
\emph{interdependent} and that things happening in one country have an impact on
surrounding countries. Theories describing this in similar words generally go
under the names of \emph{policy transfer}, or -\emph{convergence} (Gilardi
2011). Many scientists state that this behavior is nothing new. For instance
Frank Dobbin \textit{et. al.} (2007) looks all the way back at the signing of
the Treaty of Westphalia for signs of policy diffusion. But they also state that
now the process of diffusion happens much faster and is spreading geographically
wider than ever before.


The main reason why there still exist so many names for almost the same process
is that until now, few mechanisms have been clearly identified as the motors of
diffusion (see Gilardi 2010, Nowlin 2011, Dobbin \textit{et. al.} 2007) and no
theory has clearly shown to be more useful than others. Many diffusion-theorists
especially criticize the lack of coordination between the different social
sciences (Savage 1985), (Katz 1999), (Dobbin \textit{et. al.} 2007) and although
a large \enquote{wave} new of studies with theoretical improvements have
appeared only since 2006 (Gilardi 2010) the mechanisms are still somewhat
\enquote{blurred}, to use words of Frank Dobbin and his group (2007).


But concentrating only on the specific string of diffusion research, there are a
number of mechanisms recognized, namely \emph{competition},
\emph{coercion}, \emph{social emulation} and \emph{learning}
(Covadonga \& Gilardi 2009). Problematic is, that they are laid out differently
in the works of different researchers and that they can happen simultaneously.
To understand how they work and what difficulties may be arising, they will be
highlighted below. 


We will begin with \emph{learning}, which by some theorists is understood
either as \emph{rational} -- or \emph{bounded} learning or understood as
including parts of both. When \emph{rational}, leaders make informed choices
about costs and benefits of policies and their effects. Frank Dobbin
\textit{et. al.} 2007, state that policy makers often have to
make assumptions about possible outcomes and rarely can base them on hard facts
because reliable results might not yet exist. This makes it hard to ever make
rational decisions. In the case of \emph{bounded learning},
policy makers use cognitive shortcuts, making learning strongly biased
(Covadonga \& Gilardi 2009). According to Fabrizio Gilardi (2010) it has to be
considered that we all are drawing on old knowledge when processing new
information and that we so are biased from the beginning, considering our
individual ideological views, values and attitudes towards policies. 


The difference between \emph{learning} and \emph{emulation} is that in the
second case, policies spread across countries only because they have become
socially accepted and not because their performance is well documented
(Covadonga \& Gilardi 2010). This happens when no real learning process is
involved and, instead of acceptance or internationalization, only imitation
takes place. According to Frank Dobbin \textit{et. al.}
(2007), many constructivist theorists argue that identification with others can
initiate emulation processes when countries are considering themselves to be
i.e. \enquote{liberal} or \enquote{progressive} and feeling a psychological or
emotional closeness with similar- or neighboring countries. This means that
when a few countries in the peer-group introduce a new policy, others shortly
follow suit. Some of the most recognized IO's, as the UN, are documented to
have this effect on countries considering themselves \enquote{moral}, which leads
them to signing treatises without much consideration.  


\emph{Competition} and \emph{coercion} on the other hand are different.
Meseguer Covadonga \& Fabrizio Gilardi (2009) state that \emph{competition}
occurs over similar and scarce resources. Because of fear of economic
disadvantages, they adopt policies to stay competitive. Frank Dobbin
\textit{et. al.} (2007) say that in order to attract global
investors countries are pushed to adopt many \enquote{market-friendly} policies,
which may not be the best road to take for that specific country. Sometimes
this can lead to a \enquote{race to the bottom- process}. 


\emph{Coercion} can be used by IO's, NGO's, or governments over weaker
governments through either physical force or manipulation of costs or benefits
(Dobbin \textit{et. al.} 2007). Often mentioned to use
coercion are the big financial institutions of the IMF, World Bank etc.
(Simmons 2000, Dobbin \textit{et. al.} 2007). They are in the
position to put extensive demands onto loans as conditions for funding. 

\section{Comparison and discussion}

\subsection{Comparing both theories}


After laying out the basics of both theories, a discussion of their usefulness
is in place. In the view of the author, two main arguments present themselves.
Firstly, they are fairly similar. Secondly, diffusion theory is still not much
consolidated as a theory. 


What are the differences between the two? Diffusion theory has been on the rise
in recent years, having scientists presenting large amounts of new articles in
an unlimited numbers of research fields. The scope of the research also varies
a lot within its research field, even if we limit ourselves to pure
policy-making diffusion in political science. Why is this? Diffusion can be
very easily measured, using the right amount of statistical data and with help
of quantitative (cross-national) analysis (Elkins \& Simmons 2005). There is a
large amount of research focusing only on the diffusion of policies within the
federal structures of countries, especially policy diffusion between states in
the U.S. Then there is diffusion between different countries, often about
welfare state policies i.e. pension systems, health care and tax systems. 


Because what you do is measure empirical change and then try to analyze why it
happened and what factors influenced the results. This gives diffusion theory a
scientific, mathematical touch, which is strongly appreciated by the scientific
community. 


On the other hand, socialization theory uses a much more qualitative approach.
When analyzing socialization, the attitudes of individuals has to measured.
Also, there are difficulties tracing the \enquote{life cycle} of norms in Table 1.
Who are the actors? What drives them? There is also no real possibility to
identify the first appearance of norms or in which stage a norm might find
itself in without interviews or media content analysis. In the case of the
studies in international socialization in this research paper, methods of
identifying socialization have included survey-studies (interviews) and
statistical regression-analysis of their data (Hooghe 2005), statistical data
on changes in spending, and interviews of UNESCO-officials (Finnemore 1993),
interviews of policymakers and politicians and in-depth studies of the
policy-processes (Kelly 2004). Compared to diffusion-studies is socialization
research a more personnel- and cost- exhaustive research approach. 


Considering the scientific advantages of purely rational models, only accepting
preferences based on mathematically measurable choices such as rational choice
behavior models, it is understandable that measuring choices and preferences
based on norms is considered scientifically not (as) adequate. But some
researchers are moving away from overly reduced models of behavior such as the
\emph{homo economicus}. Martha Finnemore (1996) has also
contributed to this literature with different texts about sociological
institutionalism. Or see Rosemary C., R. Hall \& Peter. A. Taylor (1996) for a
good overview. 


Another complicated question is the difference between diffusion/socialization
and \emph{development}, because development happens even if we cannot explain
it and this has always been the case. This has for example been brought up by a
number of researchers like Robert L. Savage (1985), Elihu Katz (1999), Lars
Carlsson (2000), Zachary Elkins and Beth Simmons (2005) but according to them
the discussion is lacking within the broader research. The reasons for change
are as many as they can be, and trying to put them under a common denominator
is not an easy task because of the sheer complexity of society. 


According to Meseguer Covadonga \& Fabrizio Gilardi (2009) is the diffusion
literature also marked by a selection bias from seeing diffusion in different
fields and then analyzing it, especially in fields where policies have
increased explosively. In addition to this, few scientists test for other
mechanisms being the reason for diffusion using the falsification approach by
Karl Popper. This may lead to a backward analysis of diffusion where only proof
for it exists whereas proof against it is neglected. 


What do they have in common? Still, they both focus on describing social and
cultural change, trying to analyze change in \emph{behavior}, \emph{norms}
and \emph{policies}. When thinking deeply about these three words, it gets
clear that in modern society none of them really can be said to come first in.
All of them can initiate a change in the other, see Table 2. 




Both theories recognize personal attitudes to influence our choices. A good
example of this is Fabrizio Gilardi (2009: 652), in his paper on
policy-learning, where he wanted to show different levels of reception to policy
change and used box-plots to depict prior vs. posterior policy-beliefs.  At the
same time, socialization- theorist Liesbet Hooghe (2005: 877) displays an almost
identical graph showing which factors are most important for socialization in
personal attitudes toward supranationalism. In his paper, Fabrizio Gilardi also
defines learning \blockquote[Gilardi 2009 drawing on a definition from Dobbin
\textit{et. al.} 2007]{as a process whereby policy makers change their beliefs
about the effects of policies}. Without any hesitation, this can be said to be
almost exact the same as socialization.


Also the other mechanisms of policy diffusion, competition, coercion or
emulation, can be used one-to-one in socialization theory. If this is so, why
do even two different models exist? Lars Carlsson (2000: 201), quoting Harold
Lasswell (1968), states, that a new aim of policy-making studies emerged in the
1950's, where not only a descriptive study was performed, but where the goal
was to improve on the policy-making process in itself. Before that it was not
really accepted for scientists to make assumptions about where the future
should be heading. Instead this belonged to political philosophy or ethics, or
it was supposed to be drawing only on the opinion of voters in a more pure
bottom-up process (Carlsson 2000). Martha Finnemore (1998) tells a similar
story, where the normative and empirical worlds should have been kept apart,
meaning that normative claims had no place in policies. This seems like a
reason for the division between the two theories, where socialization became
unfashionable over almost 30 years and policy-diffusion theory gradually went
from neutrally measuring change in policies to identifying mechanisms close to
socialization.


Today it is clear that norms have to be researched as an essential part of
IO's because conditions for membership (and trade) are not only concerning
practical aspects, instead they often include clauses on normative behavior as
well. But also when, according to Ian Hurd (2010), formal and informal rules
get more important. Or when treatises with both hard and soft law are used
(Abbott und Snidal 2000). Countries know, that when gaining membership of
highly normative organizations, for example the EU, and/or an organization
equipped with great economic power like the World Bank or the IMF, that these
organizations \emph{can}, and, in the eyes of many organizations and
countries, \emph{even should}, use norms to quicken the
pace of \emph{development}. Or, using other words, the pace of socialization
or learning-process.

\subsection{Can IO's lead to peace? }


As a last step in this research paper, the question about socialization- and
diffusion theory and their potential role in promoting peace is to be analyzed.
First, some underlying mechanics of the international scene will be examinated.
This is important for the second goal of the paper, to make sense of some of
the international structures on the international scene.  


Going back to one of the first question about the dynamics between countries:
Why do states cooperate with each other? Ian Hurd (2011) drawing on Jan
Klabbers (2002) describes three generally acknowledged features of an IO,
IO's as \emph{actors}, \emph{forum} and \emph{resources}. A
short summary from the works of Ian Hurd (2010), James Fearon (1998) and
Kenneth W. Abbot \& Duncan Snidal (1998) concerning cooperation is given in the
section below: 


Countries have problems that they would like to solve, but lack the resources
themselves so they pool their resources and create a new actor, a collective
unit which can draw on the resources to achieve its goals. When designing the
treatises for the new IO, it can be fitted with varying amounts of independence
and sanctioning power. They also create a new place, a forum, where members can
discuss difficulties arising between them and the world outside. This has the
possibility to reduce some of the effects of differences in size and wealth
between members and to give smaller countries more leverage in group.  It also
gives informational advantages inside as well as reduces uncertainties between
members because of norms of behavior and written rules and regulations.  


Liberalism states that IO's foster peace between countries, which has been
confirmed by some scientists (see Abbot and Snidal 1998) and denied by others.
The deniers, often realists, are saying that IO's \blockquote[Boehmer
\textit{et. al.} 2004][]{reflect, rather than effect, world politics}. Whatever
the case might be, IO's play an important role designing treaties and fostering
new norms on the international scene. Solving the bargaining problem of
different preferences over IO treaties and the following monitoring and
enforcement problem (Fearon 1998) is important for international relations
between states and organizations. 


But the question is, are they right to force norms onto countries? Norms more
often than not include parts of religious values or values of institutional
design and efficiency which non-western countries do not share. 


Some of these norms have become internalized and are not even noticed by
countries or citizens any more. For example, new legal professionals design
treaties in certain ways according to professional norms applying to specific
areas of interest. The way they do this is dependent of the education they have
received, both inside and outside their native countries and organizations. For
instance in the case of the UNESCO (see Martha Finnemore 1994), it gets clear
that officials were involved in shaping science institutions in many countries
not only as advisers but as driving forces in designing institutions. The same
paper also shows the question if the UNESCO officials were right to do as they
did and shows the evolution of norms concerning direct involvement over time. 


Why do countries sign up for membership in IO's (that they don't need?)?
Kenneth Abbot and Duncan Snidal (1998) answer this with that \enquote{the role of
IO's extends even further to include the development of common norms and
practices that help define, or refine states themselves}. This statement is
drawing on constructivism theory that supports parts of socialization- as well
as diffusion- theory. It shows that identities and reputations get assessed in-
and between countries with the help of IO membership (Hurd 2010).


Why do they comply? Classical rational choice theorists would argue that
countries comply as long as the trade-off between wins and losses are good
enough. A big and powerful country such as the United States has less to lose
by defecting (and should therefore do so occasionally according to Brooks and
Wohlforth 2005), than a small country with less resources and bargaining power
on the international scene. Beth Simmons (2000) answers some of these questions
with reputational costs of defecting. By becoming members they show credibility
and might gain competitive advantages through information or resources, but
they also signal to foreign investors that they are \enquote{trustworthy}. Simmons
further conclude that reputational assessments rather exist on \enquote{markets}
than measured only within IO's and that more de-centralized forces are
pushing for compliance than IO's. Ian Hurd (2010) also supports this, saying
that regimes of formal and informal rules spread through regions and lead to
clustering of countries. 


This supports the claims made by Robert D. Putnam (1998) of the
\enquote{two-level}-game. In each country the cultural and political contexts
make a difference and have to be considered. For example, because of
reputational costs, no leaders of today in a western democracy could even
propose to leave big Institutions like the UN or NATO. Or even if they did, as
when smaller parties in European countries are discussing leaving the EU, they
do not have the majority of citizens behind them. Many citizens have been
socialized into believing that the structure of existing IO's and their
memberships is beneficial for them. 


On the opposite side, many times there is no real acceptance of IO norms made by
Institutions such as the World Bank within its receiving third world countries,
which would be essential for actually promoting changes and learning from
ideas. As a result only copying or emulation takes place because of the
perceived coercion. Or anti-opinions are formed in people and politicians.  This
type of learning process is included in the diffusion theory, which also
controls for learning the \blockquote[Gildardi 2010]{wrong lessons}, but not in
the same extent in socialization-theory.


All of these arguments are important for IO's when designing new treaties.
Finding out how they can influence member states, through which methods of hard
and soft law does the author believe to be crucial for development in the
world. Finding out why international cooperation does not work in some cases
and where it does work well could be researched using both socialization-based
methods as well as diffusion theory. This could lead to new priorities and more
acceptance in third world countries as well as more efficient funding within
institutions.  

\section{Summary}

In this paper two theories have been analyzed and compared, namely international
socialization- and diffusion-theory. 


International socialization theory is using norm-based theory when analyzing
changes in behavior in countries or Organizations. It uses norms about behavior
to see where IO's have used positive or negative sanctions on actors and so
enforced certain behavior, and trying to see in which cases this has been
successful. Researching the change in norms and behavior has to be done mostly
through qualitative research, undertaking interviews but also with the help of
using statistical models. 


Policy diffusion analyzes diffusion of policies between countries or states in a
federal system. It recognizes that the world is interdependent and what happens
in one country has an effect on the citizens and policy-makers in another
country. Four mechanisms have been identified as pushing policy diffusion,
namely learning, social emulation competition and coercion. Policy diffusion
studies uses statistical data and cross-country analysis to find proof for
which mechanisms have been at work in different countries and if general trends
can be found. 


Both theories are using very similar indicators of measuring social and cultural
change and sometimes feel like the same method divided into one qualitative and
one quantitative method. This is supported by the historical context in which
both theories have been developed in, which happened in parallel. Also, both
theories have strong roots in the other social sciences such as sociology and
psychology. For an outsider, the sheer amount of diffusion studies can be a
misleading indicator to think that diffusion is a technically better and
consolidated research theory. This proved not to be the case. Diffusion theory
theoretically has still some road ahead of itself before a broad acceptance
about which mechanisms of diffusion should be applied and how. But because of
its usability in every policy-area and together with its quantitative approach,
it gives a clear indicator to why so much of the research took place in the
field of diffusion instead of socialization.


In the end they are both theories about preferences, attitudes and beliefs. It
is what everyday life is about, as well as politics. Considering normative
behavior in countries or policy-makers must not stand against rational
behavior-models. Instead they should be combined. But also un-rational behavior
has to be considered. This fact is recognized by diffusion-theorists where
mechanisms can show proof of learning from false things or showing emulation of
behavior instead of internalizing reasons. 


If current research on IO's continues down this road, drawing on conclusions
from scientists in the variety of themes in the international relations
research such as institutional design, cooperation, legalization, compliance,
etc. maybe IO's can promote peace on a larger scale. Both theories are good
tools for analyzing trends and clustering of countries or acceptance of
behavior. But the importance of norms cannot be ignored, on the international
scene as well as domestically. 

\clearpage

\section*{References} % hier verwende ich die "starred" Variante des
% section Kommandos, da das Literaturverzeichnis selber nie im
% Inhaltsverzeichnis steht und auch nicht nummeriert werden sollte

\singlespacing


\begin{list}{}{\setlength{\leftmargin}{1.5em}\setlength{\itemindent}{-1.5em}\setlength{\itemsep}{0.5ex}}

\item ABBOT, Kenneth and SNIDAL, Duncan (1998): \enquote{Why States Act through
Formal International Organizations?}  \textit{Journal of Conflict Resolution}
42(1): p. 3--32.

\item ABBOT, Kenneth and SNIDAL, Duncan (2000): \enquote{Hard and Soft Law in
International Governance?} \textit{International Organization} 54(3): p.
421--456.

\item ABBOT, Kenneth, KEOHANE, Robert, O., MORAWCSIK, Andrew, SLAUGHTER,
Anne-Marie and SNIDAL, Duncan (2000): \enquote{The Concept of
Legalization}. \textit{International Organization} 54(3): p.  401--419.

\item AXELROD, Robert (1981): \enquote{The Emergence of Cooperation among
Egoists}. \textit{American Political Science Review} 75(2): p. 306--318.

\item BOEHMER, Charles, GARTZKE, Erik and NORDSTROM, Timothy (2004): \enquote{Do
Intergovernmental Organizations promote Peace?} \textit{World Politics} 57(1):
1--38.

\item BROOKS, Stephan G. and WOHLFORTH, William, C.  (2005):
\enquote{International Relations Theory and the Case against Unilateralism}.
\textit{Perspective on Politics} 3(3): p. 509--524.

\item CARLSSON, Lars (2000): \enquote{Non-Hierarchical Evaluation of
Policy}. \textit{Evaluation} 6(2): p.  201-- 216.

\item COVADONGA, Meseguer \& GILARDI, Fabrizio (2009): \enquote{What is new in
the study of policy diffusion}. \textit{Review of International Political
Economy} 16(3): p 527--543.

\item DENNIS, Jack (1968): \enquote{Major Problems of Political Socialization
Research}. \textit{Midwest Journal of Political Science} 12(1): p. 85--114.

\item DOBBIN, Frank, SIMMONS, Beth and GARETT, Geoffrey (2007): \enquote{The
Global Diffusion of Public Policies: Social Construction, Coercion, Competition,
or Learning?} \textit{Annual Review of Sociology} 33: p.  449--472.

\item ELKINS, Zachary and SIMMONS, Beth (2005): \enquote{On Waves, Clusters, and
Diffusion: A Conceptual Framework}.  \textit{Annals of the American Academy of
Political and Social Science} 2005: p. 33--51.

\item FEARON, James, D. (1998): \enquote{Bargaining, Enforcement, and
International Cooperation}. \textit{International Organization} 52(2): p.
269--305.

\item FINNEMORE, Martha (1993): \enquote{International Organizations as Teachers
of Norms: The United Nations Educational, Scientific, and Cultural Organization
and Science Policy}. \textit{International Organization} 47(4): p. 565--597.

\item FINNEMORE, Martha. (1996): \enquote{Norms, culture, and world politics:
insights from sociology's institutionalism}. \textit{International Organization}
50(2): p. 325--347.

\item FINNEMORE, Martha \& SIKKINK, Kathryn (1998): \enquote{International Norm
Dynamics and Political Change}.  \textit{International Organization} 54(4):
p. 887--917.

\item GILARDI, Fabrizio (2010): \enquote{Who Learns from What in Policy
Diffusion Process}.  \textit{American Journal of Political Science} 54(3): p.
650--666.

\item GILARDI, Fabrizio (2011): \enquote{Policy interdependence: transfer,
diffusion, convergence}. University of Zuerich. Working paper - First draft. May
18th 2011.

\item HALL, Peter. A. and TAYLOR, Rosemary, C., R.  (1996): \enquote{Political
Science and the Three New Institutionalisms}. \textit{Political Studies} 44:
p. 936--957.

\item HOOGHE, Liesbet (2005): \enquote{Several Roads Lead to International
Norms, but Few via International Socialization: A Case Study of the European
Commission}. \textit{International Organization} 59(4): p. 861--898.

\item HURD, Ian (2010): \enquote{A Guide to the Study of International
Organizations: Politics, Law, Practice}. Cambridge: Cambridge University Press.

\item KATZ, Elihu (1999): \enquote{Theorizing Diffusion: Tarde and Sorokin
Revisited}. \textit{Annals of the American Academy of Political and Social
Science} 566: p. 144--155.

\item KELLY, Judith (2004): \enquote{International Actors on the Domestic Scene:
Membership Conditionality and Socialization by International
Organizations}. \textit{International Organization} 58(3): p. 425--457.

\item KEOHANE, Robert, O. (1986): \enquote{Reciprocity in International
Relations}. \textit{International Organization} 40(1): p.1--27.

\item MORROW, James, D. (1994): \enquote{Modeling the forms of international
cooperation: distribution versus information}.  \textit{International
Organization} 48(3): p 387--423.

\item NOWLIN, Matthew C. (2011):\enquote{Theories of the Policy Process: State
of the Research and Emerging Trends}. \textit{Policy Studies Journal} 39(1):
p. 41--60.

\item PUTNAM, Robert, D. (1998): \enquote{Diplomacy and domestic politics: the
logic of two-level games}. \textit{International Organization} 42(3):
p. 427--460.

\item SAVAGE, Robert L. (1985): \enquote{Diffusion Research Traditions and the
Spread of Policy Innovations in a Federal System}. \textit{Policy Diffusion in a
Federal System} 15(4): p. 1--27.

\item SIMMONS, Beth, A. (2000): \enquote{International Law and State Behavior:
Commitment and Compliance in International Monetary Affairs}. \textit{American
Political Science Review} 94(4): p.  819--835.

\item WILSON, Richard, W. (1981): \enquote{Political Socialization and Moral
Development}. \textit{World Politics} 33(2): p.  153--177.

\end{list}


% \begin{thebibliography}{99}

% \bibitem{abbot1998} ABBOT, Kenneth and SNIDAL, Duncan (1998):
% \enquote{Why States Act through Formal International Organizations?}
% Journal of Conflict Resolution 42(1): p. 3--32.

% \bibitem{abbot2000a} ABBOT, Kenneth and SNIDAL, Duncan (2000):
% \enquote{Hard and Soft Law in International Governance?} International
% Organization 54(3): p.  421--456.

% \bibitem{abbot2000b} ABBOT, Kenneth, KEOHANE, Robert, O., MORAWCSIK,
% Andrew, SLAUGHTER, Anne-Marie and SNIDAL, Duncan (2000): \enquote{The
% Concept of Legalization}. International Organization 54(3): p.
% 401--419.

% \bibitem{axelrod1981} AXELROD, Robert (1981): \enquote{The Emergence
% of Cooperation among Egoists}. The American Political Science Review
% 75(2): p. 306--318.

% \bibitem{boehmer2004} BOEHMER, Charles, GARTZKE, Erik and NORDSTROM,
% Timothy (2004): \enquote{Do Intergovernmental Organizations promote
% Peace?} World Politics 57(1): 1--38.

% \bibitem{brooks2005} BROOKS, Stephan G. and WOHLFORTH, William, C.
% (2005): \enquote{International Relations Theory and the Case against
% Unilateralism}.  Perspective on Politics 3(3): p. 509--524.

% \bibitem{carlsson2000} CARLSSON, Lars (2000):
% \enquote{Non-Hierarchical Evaluation of Policy}. Evaluation 6(2): p.
% 201-- 216.

% \bibitem{covadonga2009} COVADONGA, Meseguer \& GILARDI, Fabrizio
% (2009): \enquote{What is new in the study of policy diffusion}. Review
% of International Political Economy 16(3): p 527--543.

% \bibitem{dennis1968} DENNIS, Jack (1968): \enquote{Major Problems of
% Political Socialization Research}. Midwest Journal of Political
% Science 12(1): p. 85--114.

% \bibitem{dobbin2007} DOBBIN, Frank, SIMMONS, Beth and GARETT, Geoffrey
% (2007): \enquote{The Global Diffusion of Public Policies: Social
% Construction, Coercion, Competition, or Learning?} Annual review of
% Sociology 33: p.  449--472.

% \bibitem{elkins2005} ELKINS, Zachary and SIMMONS, Beth (2005):
% \enquote{On Waves, Clusters, and Diffusion: A Conceptual Framework}.
% The annals of the American Academy of Political and Social Science
% 2005: p. 33--51.

% \bibitem{fearon1998} FEARON, James, D. (1998): \enquote{Bargaining,
% Enforcement, and International Cooperation}. International
% Organization 52(2): p.  269--305.

% \bibitem{finnemore1993} FINNEMORE, Martha (1993):
% \enquote{International Organizations as Teachers of Norms: The United
% Nations Educational, Scientific, and Cultural Organization and Science
% Policy}. International Organization 47(4): p. 565--597.

% \bibitem{finnemore1996} FINNEMORE, Martha. (1996): \enquote{Norms,
% culture, and world politics: insights from sociology's
% institutionalism}. International Organization 50(2): p. 325--347.

% \bibitem{finnemore1998} FINNEMORE, Martha \& SIKKINK, Kathryn (1998):
% \enquote{International Norm Dynamics and Political Change}.
% International Organization 54(4): p. 887--917.

% \bibitem{gilardi2010} GILARDI, Fabrizio (2010): \enquote{Who Learns
% from What in Policy Diffusion Process}.  American Journal of Political
% Science 54(3): p.  650--666.

% \bibitem{gilardi2011} GILARDI, Fabrizio (2011): \enquote{Policy
% interdependence: transfer, diffusion, convergence}. University of
% Zuerich. Working paper - First draft. May 18th 2011.

% \bibitem{hall1996} HALL, Peter. A. and TAYLOR, Rosemary, C., R.
% (1996): \enquote{Political Science and the Three New
% Institutionalisms}. Political Studies 44: p. 936--957.

% \bibitem{hooghe2005} HOOGHE, Liesbet (2005): \enquote{Several Roads
% Lead to International Norms, but Few via International Socialization:
% A Case Study of the European Commission}. International Organization
% 59(4): p. 861--898.

% \bibitem{hurd2010} HURD, Ian (2010): \enquote{A Guide to the Study of
% International Organizations: Politics, Law, Practice}. Cambridge:
% Cambridge University Press.

% \bibitem{katz1999} KATZ, Elihu (1999): \enquote{Theorizing Diffusion:
% Tarde and Sorokin Revisited}. Annals of the American Academy of
% Political and Social Science 566: p. 144--155.

% \bibitem{kelly2004} KELLY, Judith (2004): \enquote{International
% Actors on the Domestic Scene: Membership Conditionality and
% Socialization by International Organizations}. International
% Organization 58(3): p. 425--457.

% \bibitem{keohane1986} KEOHANE, Robert, O. (1986): \enquote{Reciprocity
% in International Relations}. International Organization 40(1): p.1--27.

% \bibitem{morrow1994} MORROW, James, D. (1994): \enquote{Modeling the
% forms of international cooperation: distribution versus information}.
% International Organization 48(3): p 387--423.

% \bibitem{nowlin2011} NOWLIN, Matthew C. (2011):\enquote{Theories of
% the Policy Process: State of the Research and Emerging Trends}. Policy
% Studies Journal 39(1): p. 41--60.

% \bibitem{putnam1998} PUTNAM, Robert, D. (1998): \enquote{Diplomacy and
% domestic politics: the logic of two-level games}. International
% Organization 42(3): p. 427--460.

% \bibitem{savage1985} SAVAGE, Robert L. (1985): \enquote{Diffusion
% Research Traditions and the Spread of Policy Innovations in a Federal
% System}. Policy Diffusion in a Federal System 15(4): p. 1--27.

% \bibitem{simmons2000} SIMMONS, Beth, A. (2000): \enquote{International
% Law and State Behavior: Commitment and Compliance in International
% Monetary Affairs}. American Political Science Review 94(4): p.
% 819--835.

% \bibitem{wilson1981} WILSON, Richard, W. (1981): \enquote{Political
% Socialization and Moral Development}. World Politics 33(2): p.
% 153--177.

% \end{thebibliography}

\end{document}
